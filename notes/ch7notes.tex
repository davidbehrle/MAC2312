\documentclass[a4paper]{article}
\usepackage[margin=0.5in]{geometry}
\usepackage[english]{babel}
\usepackage[utf8]{inputenc}
\usepackage{textcomp}
\usepackage{amsmath}
\usepackage{gensymb}
\usepackage{xcolor}
\usepackage{array}
\usepackage{tabularx}
\usepackage{tikz}
\usepackage{pgfplots}
\usepackage{framed}
\usepackage{xfrac}
\usepackage[most]{tcolorbox}
\usepackage{fix-cm}
\usetikzlibrary{quotes,angles}
\usetikzlibrary{calc}
\usepgfplotslibrary{fillbetween}

\let\bf\textbf
\colorlet{shadecolor}{orange!15}
\pgfplotsset{compat=1.18}
\newcommand\der[2]{\frac{d #1}{d #2}}
\newcommand\Deltax{\Delta x}
\newcommand{\AxisRotator}[1][rotate=0]{%
    \tikz [x=0.25cm,y=0.60cm,line width=.2ex,-stealth,#1] \draw (0,0) arc (-150:150:1 and 1);%
}
\def\centerarc[#1](#2)(#3:#4:#5){\draw[#1] ($(#2)+({#5*cos(#3)},{#5*sin(#3)})$) arc (#3:#4:#5)}
% Syntax: \centerarc[draw options] (center) (initial angle:final angle:radius);

\title{Techniques of Integration}
\author{Calculus: Early Transcendentals 9e}
\date{}

\begin{document}
\maketitle
\setcounter{section}{7}
\subsection{Integration by Parts}
Recall the product rule:
\begin{align*}
    \der{}{x}\big[f(x)g(x)\big] = f(x)g'(x) + g(x)f'(x)
\end{align*}
Written as an integral:
\begin{align*}
    \int\big[f(x)g'(x) + g(x)f'(x)\big]dx = f(x)g(x)
\end{align*}
or
\begin{align*}
    \int f(x)g'(x)dx + \int g(x)f'(x)dx = f(x)g(x)
\end{align*}
This can be rearranged to:
\begin{equation}
    \int f(x)g'(x)dx = f(x)g(x) - \int g(x)f'(x)dx
\end{equation}
This is the equation for integration by parts. To make it easier to read, let $u = f(x)$ and $v = g(x)$. The differentials are $du = f'(x)dx$ and $dv = g'(x)dx$. This makes the formula for integration by parts:
\begin{equation}
    \int u\;dv = uv - \int v\;du
\end{equation}
\begin{shaded}
    \noindent\bf{Example 1}
    \vspace{2mm}\\
    Find $\displaystyle\int x\sin(x)dx$
    \begin{align*}
        u = x \hspace{10mm} dv = \sin(x)dx
    \end{align*}
    Take the derivative of $u$ and antiderivative of $dv$, therefore:
    \begin{align*}
        du = dx \hspace{10mm} v = -\cos(x)
    \end{align*}
    Using formula 1:
    \begin{align*}
        \int x\sin(x)dx &= -x\cos(x) + \int \cos(x)dx\\
        &= -x\cos(x) + \sin(x) + C
    \end{align*}
\end{shaded}
\noindent In general, the function chosen to be $u = f(x)$ is one that becomes simpler or at least not more complicated when differentiated. $u$ can be determined using the priority list:
\begin{center}
    \begin{tabularx}{0.3\textwidth}{
        | >{\raggedright\arraybackslash}X
        | >{\raggedright\arraybackslash}X | }
        \hline
        Logarithmic & $\ln(x)$\\
        \hline
        Inverse Trig & $\tan^{-1}(x)$\\
        \hline
        Algebraic & $5x^2 + 3$\\
        \hline
        Trigonometric & $\cos(x)$\\
        \hline
        Exponential & $10^x$\\
        \hline
    \end{tabularx}
\end{center}

\pagebreak

\begin{shaded}
    \noindent\bf{Example 2}
    \vspace{2mm}\\
    Find $\displaystyle\int \ln(x)dx$
    \begin{align*}
        u &= \ln(x) \hspace{10mm} dv = dx\\
        du &= \frac{1}{x} \hspace{16.5mm} v = x
    \end{align*}
    Therefore:
    \begin{align*}
        \int \ln(x)dx &= x\ln(x) - \int x\bigg(\frac{1}{x}\bigg)dx\\
        &= x\ln(x) - x + C
    \end{align*}
\end{shaded}

\begin{shaded}
    \noindent\bf{Example 3}
    \vspace{2mm}\\
    Find $\displaystyle\int t^2e^t dt$
    \begin{align*}
        u &= t^2 \hspace{10mm} dv = e^tdt\\
        du &= 2tdt \hspace{8.5mm} v = e^t
    \end{align*}
    Therefore:
    \begin{align*}
        \int t^2e^tdt = t^2e^t - 2\int te^tdt
    \end{align*}
    The integral $\displaystyle\int te^tdt$ also requires integration by parts
    \begin{align*}
        u_1 &= t \hspace{10mm} dv = e^tdt\\
        du_1 &= dt \hspace{10mm} v = e^t
    \end{align*}
    Therefore:
    \begin{align*} 
        \int t^2e^tdt &= t^2e^t - 2\Big(te^t - \int e^tdt \Big)\\
        &= t^2e^t - 2te^t + 2e^t
    \end{align*}
\end{shaded}
\end{document}